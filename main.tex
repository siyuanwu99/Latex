\documentclass[aspectratio=169, 13pt,compress]{beamer}
\hypersetup{pdfpagemode=FullScreen}
\mode<presentation>
%{
%	\usetheme{WSY}
%}
% including all the requirements in this project
% copied from https://www.overleaf.com/latex/templates/dead-physicists-society-presentation-template/zqmtrkmgxzqz


\usepackage{amsthm}
\usepackage{mathtools}
\usepackage{physics}  % add physics support
\usepackage{csquotes}  % inline and display quotations
%\usepackage{tensor}  % add tensor support for math
\usepackage[thicklines]{cancel}
\usepackage{tcolorbox}
%\usepackage{pstricks}
\usepackage{multicol}
\usepackage{xeCJK}
\usepackage{fontspec}
\usepackage{listings}
\usepackage{xcolor}
\usepackage{sourcecodepro}
\usepackage{hyperref}
\usepackage{tikz}
\usepackage{minted}


\renewcommand{\CancelColor}{\color{royalblue}}


%%%%%  Highlights


%%%%% Math symbols
\newcommand{\scriptr}{\mathcalligra{r}\,}
\newcommand{\boldscriptr}{\pmb{\mathcalligra{r}}\,}
\def\rc{\scriptr}
\def\brc{\boldscriptr}
\def\hrc{\hat\brc}
\newcommand{\ie}{\emph{i.e.}} %id est
\newcommand{\eg}{\emph{e.g.}} %exempli gratia
\newcommand{\rtd}[1]{\ensuremath{\left\lfloor #1 \right\rfloor}}
\newcommand{\dirac}[1]{\ensuremath{\delta \left( #1 \right)}}
\newcommand{\diract}[1]{\ensuremath{\delta^3 \left( #1 \right)}}
\newcommand{\e}{\ensuremath{\epsilon_0}}
\newcommand{\m}{\ensuremath{\mu_0}}
\newcommand{\V}{\ensuremath{\mathcal{V}}}
\newcommand{\prnt}[1]{\ensuremath{\left(#1\right)}} %parentheses
\newcommand{\colch}[1]{\ensuremath{\left[#1\right]}} %square brackets
\newcommand{\chave}[1]{\ensuremath{\left\{#1\right\}}}  %curly brackets

\makeatletter
\newcommand{\mybox}[1]{%
  \setbox0=\hbox{#1}%
  \setlength{\@tempdima}{\dimexpr\wd0+13pt}%
  \begin{tcolorbox}[colback=DeepSkyBlue,colframe=DeepSkyBlue,boxrule=0.5pt,arc=4pt,
      left=6pt,right=6pt,top=6pt,bottom=6pt,boxsep=0pt,width=\@tempdima]
    \textcolor{white}{#1}
  \end{tcolorbox}
}
\makeatother


\tikzset{
	dummy/.style    = {circle,draw}
}




\title{语音信号时域分类实验}
\author{郜炫齐$~$刘贵涛$~$刘少腾$~$潘昊璇$~$吴思源}
\date{2019年10月11日}

\begin{document}
\maketitle
% \section*{Contents}
\frame{
\centering
\tableofcontents[hideallsubsections]
}

\section{数据预处理}
\frame{\sectionpage}

\begin{frame}{加窗函数}
加窗分帧实际上是对信号做一个滤波,两种窗都可以滤去高频分量
\vspace{1cm}
 \begin{columns}
%  \uncover<1-2>
 \column{0.3\textwidth}
 
 \hlightb{ Square Window}
    $$ w(n) = \left \{ \begin{array}{cc}
        1 & 0 \leq n \leq N - 1 \\
        0 & \textit{others} 
    \end{array} \right.$$ 
% \uncover<2>
\column{0.7 \textwidth}

\hlightb{Hanning Window}

$$ w(n) = \left \{ \begin{array}{cc}
     0.5(1- \cos{\frac{2 \pi n}{N -1 }}) & 0 \leq n \leq N -1  \\
     0 & \textit{others}
\end{array}
\right. $$
 \end{columns}
 
\end{frame}{}
\begin{frame}[center]{音频处理}
\centering
    采用双门限方法剪辑音频数据,取出有效部分\par
    帧率设置为每秒 85 帧 \par
    \vspace{1.0cm}
    将采集到的每类 100 个数据打乱之后划分训练集和测试集 \par
    \vspace{0.5cm}
    \begin{tabular}{c|c|c}
        训练集 & 验证集 & 测试集 \\ \hline
         850 & 100 & 50
    \end{tabular}
\end{frame}

\frame[containsverbatim]{
\frametitle{特征提取}
\framesubtitle{提取时域序列统计特征}

\begin{minted}{python}
func = lambda x: [np.max(x), np.std(x), np.mean(x)]
feature = np.hstack([
            func(square_energy),
            func(hanning_energy),
            func(square_azrate),
            func(hanning_azrate),
            func(square_data),
            func(hanning_data),
            [upper_rate]
        ])
\end{minted}
}

\frame[containsverbatim]{
\frametitle{特征提取}
\framesubtitle{一点说明}
\centering
\begin{enumerate}
    \item 全局特征:均值、方差、极大值
    \item 局部特征:某一部分的值
\end{enumerate}{}
\vspace{2cm}
经过实验发现采用序列全局特征进行分类效果更优

}


\section{分类器选择}
\frame{\sectionpage}

\frame{
\centering
\begin{enumerate}
    \item 决策树
    \item 支持向量机
    \item 朴素贝叶斯
\end{enumerate}{}
}


\begin{frame}{决策树}
\begin{tikzpicture}
  [
    grow                    = right,
    sibling distance        = 6em,
    level distance          = 10em,
    edge from parent/.style = {draw, -latex},
    every node/.style       = {font=\footnotesize},
    sloped
  ]
  \node [root] {Feature $ (1 \times 19) $}
    child { node  {Assigned to \textbf{Class 1}}
      edge from parent node [below] { $x_1\leq 0.05 $} }
    child { node [dummy] {}
      child { node [dummy] {}
        child { node  {Assigned to \textbf{Class 5}}
          edge from parent node [below] {$x_{19} < 0.2$} }
        child { node  {Assigned to \textbf{Class 1}}
          edge from parent node [above] {$x_{18} \leq 0.35$}}
        child { node  {Assigned to \textbf{Class 8}}
                edge from parent node [above] {$0.2\geq x_{19} < 0.35$} }
        edge from parent node [below] {$x_5> 0.7$} }
      child { node {Assigned to \textbf{Class 4}}
              edge from parent node [above, align=center]
                {$x_5\leq 0.7$}}
              edge from parent node [above] {$x_1\geq 0.05$} };
\end{tikzpicture}
$$H\left(X_{m}\right)=-\sum_{k} p_{m k} \log \left(p_{m k}\right)$$
\end{frame}


\begin{frame}{支持向量机}

\centering
    \begin{tikzpicture}[&gt;=stealth']
  % Draw axes
  \draw [&lt;-&gt;,thick] (0,5) node (yaxis) [above] {$y$}
        |- (5,0) node (xaxis) [right] {$x$};
  % draw line
  \draw (0,-1) -- (5,4); % y=x-1
  \draw[dashed] (-1,0) -- (4,5); % y=x+1
  \draw[dashed] (2,-1) -- (6,3); % y=x-3
  % \draw labels
  \draw (3.5,3) node[rotate=45,font=\small] 
        {$\mathbf{w}\cdot \mathbf{x} + b = 0$};
  \draw (2.5,4) node[rotate=45,font=\small] 
        {$\mathbf{w}\cdot \mathbf{x} + b = 1$};
  \draw (4.5,2) node[rotate=45,font=\small] 
        {$\mathbf{w}\cdot \mathbf{x} + b = -1$};
  % draw distance
  \draw[dotted] (4,5) -- (6,3);
  \draw (5.25,4.25) node[rotate=-45] {$\frac{2}{\Vert \mathbf{w} \Vert}$};
  \draw[dotted] (0,0) -- (0.5,-0.5);
  \draw (0,-0.5) node[rotate=-45] {$\frac{b}{\Vert \mathbf{w} \Vert}$};
  \draw[-&gt;] (2,1) -- (1.5,1.5);
  \draw (1.85,1.35) node[rotate=-45] {$\mathbf{w}$};
  % draw negative dots
  \fill[red] (0.5,1.5) circle (3pt);
  \fill[red]   (1.5,2.5)   circle (3pt);
  \fill[black] (1,2.5)     circle (3pt);
  \fill[black] (0.75,2)    circle (3pt);
  \fill[black] (0.6,1.9)   circle (3pt);
  \fill[black] (0.77, 2.5) circle (3pt);
  \fill[black] (1.5,3)     circle (3pt);
  \fill[black] (1.3,3.3)   circle (3pt);
  \fill[black] (0.6,3.2)   circle (3pt);
  % draw positive dots
  \draw[red,thick] (4,1)     circle (3pt); 
  \draw[red,thick] (3.3,.3)  circle (3pt); 
  \draw[black]     (4.5,1.2) circle (3pt); 
  \draw[black]     (4.5,.5)  circle (3pt); 
  \draw[black]     (3.9,.7)  circle (3pt); 
  \draw[black]     (5,1)     circle (3pt); 
  \draw[black]     (3.5,.2)  circle (3pt); 
  \draw[black]     (4,.3)    circle (3pt); 
\end{tikzpicture}
\end{frame}{}

\begin{frame}{朴素贝叶斯}
这是一种基于贝叶斯定理的分类算法。基本假设是特征之间\hlight{相互独立}
\footnote{条件独立性假设} \par
\vspace{0.5cm}
 $$ {P\left(y | x_{1}, \ldots, x_{n}\right) \propto P(y) \prod_{i=1}^{n} P\left(x_{i} | y\right)}  $$
 
 $$ \hat{y}=\arg \max _{y} P(y) \prod_{i=1}^{n} P\left(x_{i} | y\right) $$
\end{frame}

\begin{frame}[containsverbatim]
\frametitle{Codes}
\framesubtitle{Talk is cheap, show me the code.}
\begin{minted}{python}
    if self.classifier == 'lsvm':
        clf = sklearn.svm.LinearSVC()
    elif self.classifier == 'ksvm':
        clf = sklearn.svm.SVC(kernel='rbf', gamma='scale')
        # kernel: ‘linear’, ‘poly’, ‘rbf’, ‘sigmoid’
        # gamma: 'auto' 'scale'
    elif self.classifier == 'dctree':
        clf = sklearn.tree.DecisionTreeClassifier()
    else:
        clf = []
        assert 0, "ERROR"
\end{minted}
\end{frame}

\begin{frame}{方法比较}
    两种方法的比较
    \vspace{0.8cm}
    \hspace{1.0cm}\begin{itemize}
        \item [SVM] 测试集上不一定收敛。数据必须是线性可分的。或者必须能够在超平面上线性可分
        \item[决策树] 一定收敛
    \end{itemize}
    
    \vspace{0.8cm}
    实验结果:在训练集上SVM类方法 比较难收敛,在验证集上表现也非常不好。但是决策树类方法在验证集上表现非常好。
\end{frame}

\begin{frame}{各种方法比较}
\begin{center}
    \begin{tabular}{l|c}
        \hlightb{方法} & \hlightb{得分} \\ \hline
        决策树 & 70 $\sim$ 80 \\ 
        支持向量机 & 60 $\sim$ 70 \\ 
        朴素贝叶斯 & 40 $\sim$ 50 \\
    \end{tabular}
\end{center}
    
\end{frame}

\section{多分类方法}
\frame{\sectionpage}

\begin{frame}{现有的多分类方法}
    \begin{itemize}
        \item One - vs - Rest  $~~\Longrightarrow~~$  正负样本不均衡的问题
        \item One - vs - One  $ ~~\Longrightarrow~~$  计算量可能过大,分类器过多
        \item Error-Correcting Output Codes
    \end{itemize}
\end{frame}

\begin{frame}{ECOC}
\framesubtitle{Error-Correcting Output Codes }
    将多分类问题转化为多个二分类问题,使用 $ m $ 个分类器去解决这 $ m $ 个问题,然后将各个分类器的结果编成一组二进制码。通过对比码表来确定最终类别。
    \vspace{0.8cm}
    \begin{block}{优点}
    有些分类器分错也能得到尽可能正确的结果
    \end{block}
    \vspace{.8cm}
    \centering{
    \hlight{但不是任意生成的二进制码都有自纠错性的}}
\end{frame}    


\begin{frame}{实验}
使用我们目前实验中表现最优的算法(决策树算法),训练40个分类器,然后选取验证集中表现最优的10个分类器进行测试,计算准确率
 \vspace{.8cm}
\begin{table}[]
\begin{tabular}{c|cccccccccl}
\cline{1-11}
分类器 & 11.  & 59.  & 29.  & 56.  & 57.  & 4.   & 16.  & 22.  & 26.  & 48.  \\ \cline{1-11}
准确率 & 0.88 & 0.88 & 0.87 & 0.86 & 0.86 & 0.84 & 0.84 & 0.84 & 0.84 & 0.84 \\ \cline{1-11}
\end{tabular}
\end{table}
 \vspace{.8cm}
    \centering{
    \hlight{但是最终准确率只有 $0.60$}}
\end{frame}{}

\begin{frame}{合适的ECOC码}
\framesubtitle{使列之间的 Hamming 距离最大,更高的容错率}
    \begin{itemize}
        \item  $ 3\leq k \leq 7$  Exhausive Codes(EC)
        \item $ 8 \leq k \leq 11$ Column Selection From Exhausive Codes(CSEC)
        \item $k > 11$ 暴力搜索$~ \Longrightarrow~$随机梯度下降方法
        \item $k>11$ BCE 方法
    \end{itemize}
\end{frame}

\begin{frame}{Exhausive Codes}
    \begin{table}[]
\begin{tabular}{llllllllllllllll}
\multicolumn{1}{c}{}   & \multicolumn{1}{c}{1} & \multicolumn{1}{c}{2} & \multicolumn{1}{c}{3} & \multicolumn{1}{c}{4} & \multicolumn{1}{c}{5} & \multicolumn{1}{c}{6} & \multicolumn{1}{c}{7} & \multicolumn{1}{c}{8} & \multicolumn{1}{c}{9} & 10 & 11 & 12 & 13 & 14 & 15 \\ \cline{2-16} 
\multicolumn{1}{c|}{1} & \multicolumn{1}{c}{1} & \multicolumn{1}{c}{1} & \multicolumn{1}{c}{1} & \multicolumn{1}{c}{1} & \multicolumn{1}{c}{1} & \multicolumn{1}{c}{1} & \multicolumn{1}{c}{1} & \multicolumn{1}{c}{1} & \multicolumn{1}{c}{0} & 0  & 0  & 0  & 0  & 0  & 0  \\
\multicolumn{1}{l|}{2} & 0                     & 0                     & 0                     & 0                     & 1                     & 1                     & 1                     & 1                     & 0                     & 0  & 0  & 1  & 1  & 1  & 1  \\
\multicolumn{1}{l|}{3} & 1                     & 1                     & 1                     & 1                     & 0                     & 0                     & 0                     & 0                     & 1                     & 1  & 1  & 0  & 0  & 0  & 0  \\
\multicolumn{1}{l|}{4} & 0                     & 0                     & 1                     & 1                     & 0                     & 0                     & 1                     & 1                     & 0                     & 0  & 1  & 0  & 0  & 1  & 1 
\end{tabular}
\end{table}
\end{frame}

\begin{frame}{最终实验结果}
\framesubtitle{序列特征 + 决策树 + ECOC码}
\centering

    \Huge Accuracy \hspace{4cm} \hlightb{0.80}
\end{frame}

\begin{frame}{参考工具箱}
\begin{itemize}
    \item 提特征$~ \Longrightarrow~$ \hlightb{tsfresh} \footnote{ \url{https://tsfresh.readthedocs.io/en/latest/}}
    \item 机器学习工具箱$~ \Longrightarrow~$ \hlightb{sci-kit learn} \footnote{ \url{https://scikit-learn.org/}}
    \item 语音信号处理$~ \Longrightarrow~$ \hlightb{librosa} \footnote{ \url{https://librosa.github.io/librosa/install.html}}

\end{itemize}
    
\end{frame}

\begin{frame}{开源地址}
    \par
    \url{https://github.com/edmundwsy/ASR-for-chinese-number}
\end{frame}


\begin{frame}
    \centering
    \Huge
    \textcolor{DoderBlue}{谢谢大家!}
\end{frame}

\end{document}
    